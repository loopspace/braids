% \iffalse meta-comment
%<*internal>
\iffalse
%</internal>
%<*readme>
----------------------------------------------------------------
braids --- a style file for drawing braid diagrams with TikZ/PGF
E-mail: stacey@math.ntnu.no
Released under the LaTeX Project Public License v1.3c or later
See http://www.latex-project.org/lppl.txt
----------------------------------------------------------------

This package defines some commands for drawing braid diagrams with TikZ/PGF.
%</readme>
%<*internal>
\fi
\def\nameofplainTeX{plain}
\ifx\fmtname\nameofplainTeX\else
  \expandafter\begingroup
\fi
%</internal>
%<*install>
\input docstrip.tex
\keepsilent
\askforoverwritefalse
\preamble
----------------------------------------------------------------
braids --- a style file for drawing braid diagrams with TikZ/PGF
E-mail: stacey@math.ntnu.no
Released under the LaTeX Project Public License v1.3c or later
See http://www.latex-project.org/lppl.txt
----------------------------------------------------------------

\endpreamble
\postamble

Copyright (C) 2011 by Andrew Stacey <stacey@math.ntnu.no>

This work may be distributed and/or modified under the
conditions of the LaTeX Project Public License (LPPL), either
version 1.3c of this license or (at your option) any later
version.  The latest version of this license is in the file:

http://www.latex-project.org/lppl.txt

This work is "maintained" (as per LPPL maintenance status) by
Andrew Stacey.

This work consists of the file  braids.dtx
and the derived files           braids.ins,
                                braids.pdf, and
                                braids.sty.

\endpostamble
\usedir{tex/latex/braids}
\generate{
  \file{\jobname.sty}{\from{\jobname.dtx}{package}}
}
%</install>
%<install>\endbatchfile
%<*internal>
\usedir{source/latex/braids}
\generate{
  \file{\jobname.ins}{\from{\jobname.dtx}{install}}
}
\nopreamble\nopostamble
\usedir{doc/latex/demopkg}
\generate{
  \file{README.txt}{\from{\jobname.dtx}{readme}}
}
\ifx\fmtname\nameofplainTeX
  \expandafter\endbatchfile
\else
  \expandafter\endgroup
\fi
%</internal>
%<*package>
\NeedsTeXFormat{LaTeX2e}
\ProvidesPackage{braids}[2011/05/07 v1.0 Tikz/PGF commands for drawing braid diagrams]
%</package>
%<*driver>
\documentclass{ltxdoc}
\usepackage[T1]{fontenc}
\usepackage{lmodern}
%\usepackage{morefloats}
\usepackage{tikz}
\usepackage{\jobname}
\usepackage[numbered]{hypdoc}
\definecolor{lstbgcolor}{rgb}{0.9,0.9,0.9} 
 
\usepackage{listings}
\lstloadlanguages{[LaTeX]TeX}
\lstset{breakatwhitespace=true,breaklines=true,language=TeX}
 
\usepackage{fancyvrb}

\newenvironment{example}
  {\VerbatimEnvironment
   \begin{VerbatimOut}[gobble=2]{example.out}}
  {\end{VerbatimOut}
   \begin{center}
%   \setlength{\parindent}{0pt}
   \fbox{\begin{minipage}{.9\linewidth}
     \lstset{breakatwhitespace=true,breaklines=true,language=TeX,basicstyle=\small}
     \lstinputlisting[]{example.out}
   \end{minipage}}

   \fbox{\begin{minipage}{.9\linewidth}
     \input{example.out}
   \end{minipage}}
\end{center}
}
\EnableCrossrefs
\CodelineIndex
\RecordChanges
\begin{document}
  \DocInput{\jobname.dtx}
\end{document}
%</driver>
% \fi
%
%
% \CharacterTable
%  {Upper-case    \A\B\C\D\E\F\G\H\I\J\K\L\M\N\O\P\Q\R\S\T\U\V\W\X\Y\Z
%   Lower-case    \a\b\c\d\e\f\g\h\i\j\k\l\m\n\o\p\q\r\s\t\u\v\w\x\y\z
%   Digits        \0\1\2\3\4\5\6\7\8\9
%   Exclamation   \!     Double quote  \"     Hash (number) \#
%   Dollar        \$     Percent       \%     Ampersand     \&
%   Acute accent  \'     Left paren    \(     Right paren   \)
%   Asterisk      \*     Plus          \+     Comma         \,
%   Minus         \-     Point         \.     Solidus       \/
%   Colon         \:     Semicolon     \;     Less than     \<
%   Equals        \=     Greater than  \>     Question mark \?
%   Commercial at \@     Left bracket  \[     Backslash     \\
%   Right bracket \]     Circumflex    \^     Underscore    \_
%   Grave accent  \`     Left brace    \{     Vertical bar  \|
%   Right brace   \}     Tilde         \~}
%
%
% \changes{1.0}{2011/05/03}{Converted to DTX file}
%
% \DoNotIndex{\newcommand,\newenvironment}
%
% \providecommand*{\url}{\texttt}
% \GetFileInfo{brqids.dtx}
% \title{The \textsf{braids} package}
% \author{Andrew Stacey \\ \url{stacey@math.ntnu.no}}
% \date{\fileversion~from \filedate}
% 
% \maketitle
% 
% \begin{center}
% \begin{tikzpicture}
% \braid[line width=3pt,line cap=round,style strand={1}{blue},number of strands=7] s_1 s_2 s_5^{-1};
% \end{tikzpicture}
% \end{center}
% 
% \section{Introduction}
% 
% This is a package for drawing braid diagrams using PGF/TikZ.
% An example follows.
% 
% %\begin{example}
% \begin{center}
% \begin{tikzpicture}[rotate=90]
% \braid[style strand={1}{red},style strand={2}{blue},style strand={3}{green}] s_1 s_2^{-1} s_1 s_2^{-1} s_1 s_2^{-1};
% \end{tikzpicture}
% \end{center}
% %\end{example}
% 
% \section{Usage}
% 
% \DescribeMacro{\braid}
% A braid is specified by the command \Verb+\braid+.
% The syntax for this command is as follows:
% 
% \Verb+\braid[style options] (name) at (coordinate) braid-word;+
% 
% The \Verb+braid-word+ is an expression in the braid group, such as \Verb+s_1 s_2^{-1}+.
% The generator labels are not significant.
% The exponent can be \Verb+1+, \Verb+{-1}+, or missing (in which case it defaults to \Verb+1+).
% Certain other symbols are allowed in the \Verb+braid-word+ which control the rendering of the braid.
% To get crossings to render at the same height, separate them with a hyphen (note: no check is made to ensure that the crossings can legally be put at the same height; \emph{caveat emptor}).
% To draw a \emph{floor}, precede the braid element by a vertical line.
% What happens then is that when the braid is rendered, the coordinates of the rectangle behind that crossing (wide enough to encompass all the strands) is passed to a command.
% The intention is that this command draw something behind the braid.
% The command is configurable by a key (see \ref{sec:styleopts}).
% 
% The (optional) \Verb+name+ acts a little like the \Verb+name+ of a TikZ node.
% When it is specified, the routine that renders the braid also saves certain coordinates as if they were node anchors.
% Specifically, \Verb+coordinate+ nodes are placed at the centre of the braid diagram and at the ends of each strand.
% The centre has the label \Verb+name+, the strands are labelled \Verb+name-number-end+ and \Verb+name-rev-number-end+, where \Verb+name+ is the name given to the braid, \Verb+number+ is the number of the strand counting from the left, and \Verb+end+ is either \Verb+s+ for the start or \Verb+e+ for the end.
% If the version with \Verb+rev+ is used then the numbers correspond to the \emph{final} positions of the braids.
% The name can also be specified with the \Verb+name+ key.
%
% The (options) \Verb+at (coordinate)+ syntax positions the braid at the \Verb+coordinate+ in the current picture.
% Due to the implementation, the coordinate has to be known at the start, but the width and height of the braid are only known at the end.
% Therefore, the braid is positioned so that the start of the first strand is at \Verb+(coordinate)+.
% This can also be specified using the \Verb+at+ key.
%
% The \Verb+style options+ set the style for the braid strands.
% They can be grouped into three types: options that set up the main parameters for the braid, options that set the default style for the strands, and options that set up styles for individual strands.
% The options are as follows.
% 
% \subsection{Style Options}
% \label{sec:styleopts}
% 
% \DescribeMacro{number of strands} The key \Verb+number of strands+ sets the minimum number of strands for the braid.
% The number of strands will grow according to the terms in the braid word so this merely sets a lower bound.
% If not set, the number of strands will be determined by the terms in the braid word.
% 
% \DescribeMacro{height}
% The key \Verb+height+ sets the height of the piece of the braid corresponding to an element in the group.
% 
% \DescribeMacro{width}
% The key \Verb+width+ sets the separation of the strands in the braid.
% 
% \DescribeMacro{border height}
% The key \Verb+border height+ adds a little extra length to the strands at the start and end of the braid.
% 
% \DescribeMacro{style strand}
% The style of the strands are controlled by two types of option.
% Style options that are set on the \Verb+\braid+ command are passed to every strand.
% It is also possible to add style options to individual strands using the key \Verb+style strand+.
% This takes two options, a strand number and a list of options to be applied to that strand.
% Thus, the syntax is \Verb+style strand={n}{options}+.
% The strands are numbered by their starting position.
% Not all of the standard TikZ style options are possible due to the way that the strands are constructed.
% Basically, the options that are allowed are those that do not require changing the path or drawing it more than once.
% 
% \DescribeMacro{floor command}
% When a floor is requested behind a crossing, the actual way to render it is determined by a command.
% This key allows the user to define that command.
% The argument to this key should be the code that should be executed for each floor.
% To avoid the hassle of getting the number of hashes right, the command should take no arguments.
% Rather, the coordinates of the rectangle are saved in to macros \Verb+\floorsx+, \Verb+\floorsy+, \Verb+\floorex+, \Verb+\floorey+ (these macros will expand to something like \Verb+10pt+) and the command should use these to position the drawing.
% The default is to draw a line at the top and at the bottom of the rectangle.
%
% \DescribeMacro{style floors}
% \DescribeMacro{style floor}
% In the spirit of separating \emph{style} and \emph{content}, the style options for the floors can be specified separately to the command (of course, they could be built in to the command).
% One advantage of this over building them in to the command is to allow them to be overridden for individual floors.
% The \Verb+style floors+ sets up options to be used for \emph{all} floors, whilst the \Verb+style floor={n}{options}+ sets up options to be used only for the \(n\)th floor.
% Anything specified in the \Verb+floor command+ will take precedence over both of these.
%
% Any other style options are passed to the underlying TikZ/PGF system and so may influence how the braid is drawn (but note that not all keys make sense due to the implementation).
%
%
% \section{Example}
%
% Here is a more detailed example.
% (Note that the floors are not drawn in the below.
% I have no idea why that is as they work when this example is put in a standalone file.
% It is presumably something to do with the documentation packages.)
%
%
% \begin{example}
% \begin{center}
% \begin{tikzpicture}
% \braid[
%   style floors={fill=yellow},
%   style floor={1}{dashed,fill=yellow!50!green},
%   floor command={%
%   \message{drawing floor}
%    \fill (\floorsx,\floorsy) rectangle (\floorex,\floorey);
%    \draw (\floorsx,\floorsy) -- (\floorex,\floorsy);
%   },
%   line width=2pt,
%   style strand={1}{red},
%   style strand={2}{blue},
%   style strand={3}{green}
% ] (braid) at (2,0) | s_1-s_3-s_5 | s_2^{-1}-s_4| s_1-s_4 s_2^{-1} s_1-s_3 s_2^{-1}-s_4^{-1};
% \fill[yellow] (2,0) circle (4pt);
% \fill[purple] (braid) circle (4pt);
% \node[at=(braid-3-s),pin=north west:strand 3] {};
% \node[at=(braid-3-e),pin=south west:strand 3] {};
% \node[at=(braid-rev-3-s),pin=north east:strand 3 (from bottom)] {};
% \node[at=(braid-rev-3-e),pin=south east:strand 3 (from bottom)] {};
% \end{tikzpicture}
% \end{center}
% \end{example}
%
% \StopEventually{}
%
% \section{Implementation}
%
% \iffalse
%<*package>
% \fi
%
% \begin{macro}{\ge@addto@macro}
% This is an expanded version of \Verb+\g@addto@macro+.
% Namely, it adds the \emph{expansion} of the second argument to the first.
%    \begin{macrocode}
\long\def\ge@addto@macro#1#2{%
  \begingroup
  \toks@\expandafter\expandafter\expandafter{\expandafter#1#2}%
  \xdef#1{\the\toks@}%
  \endgroup}
%    \end{macrocode}
% \end{macro}
% \begin{macro}{\braid}
% This is the user command.
% We start a group to ensure that all our assignments are local, and then call our initialisation code
%    \begin{macrocode}
\newcommand{\braid}[1][]{%
  \begingroup
  \braid@start{#1}}
%    \end{macrocode}
% \end{macro}
%
% \begin{macro}{\braid@process}
% This is the token swallower.
% This takes the next token on the braid specification and passes it to the handler command which decides what to do next.
%    \begin{macrocode}
\def\braid@process{%
  \afterassignment\braid@handle\let\braid@token=%
}
%    \end{macrocode}
% \end{macro}
%
% \begin{macro}{\braid@process@start}
% This is a variant of \Verb+\braid@process+ which is used at the start where we might have a few extra bits and pieces before the braid itself starts.
%    \begin{macrocode}
\def\braid@process@start{%
  \afterassignment\braid@handle@start\let\braid@token=%
}
%    \end{macrocode}
% \end{macro}
% \begin{macro}{\braid@handle@start}
% This is the handler in use at the start.
% It looks for the tokens \Verb+a+ or \Verb+(+ which (might) signal the start of an \Verb+at (coordinate)+ or \Verb+(name)+.
% If we get anything else (modulo spaces) we decide that we've reached the end of the initialisation stuff and it is time to get started on the braid itself.
%    \begin{macrocode}
\def\braid@handle@start{%
  \let\braid@next=\braid@handle
  \ifx\braid@token a
   \let\braid@next=\braid@maybe@locate
  \else
  \ifx\braid@token(%)
   \iffalse)\fi %Indentation hack!
   \let\braid@next=\braid@assign@name
  \else
  \ifx\braid@token\@sptoken
   \let\braid@next=\braid@process@start
  \fi
  \fi
  \fi
  \braid@next%
}
%    \end{macrocode}
% \end{macro}
%
% \begin{macro}{\braid@handle}
% This is the main handler for parsing the braid word.
% It decides what action to take depending on what the token is.
%    \begin{macrocode}
\def\braid@handle{%
  \let\braid@next=\braid@process
  \ifx\braid@token;
   \let\braid@next=\braid@render
  \else
  \ifx\braid@token^
   \let\braid@next=\braid@sup
  \else
  \ifx\braid@token_
   \let\braid@next=\braid@sub
  \else
  \ifx\braid@token-
   \braid@increase@levelfalse
  \else
  \ifx\braid@token[
   \let\braid@next=\braid@process@options
  \else
  \ifx\braid@token|
   \edef\braid@tmp{,\expandafter\the\value{braid@level}}%
   \ge@addto@macro\braid@floors\braid@tmp%
  \else
  \ifx\braid@token\bgroup
   \braid@beginscope
  \else
  \ifx\braid@token\egroup
   \braid@endscope
  \else
  \ge@addto@macro\braid@label{\braid@token}%
  \fi
  \fi
  \fi
  \fi
  \fi
  \fi
  \fi
  \fi
  \braid@next%
}
%    \end{macrocode}
% \end{macro}
%
% \begin{macro}{\braid@maybe@locate}
% If we got an \Verb+a+ token in the \Verb+\braid@handle@start+ then it \emph{might} mean we're looking at \Verb+at (coordinate)+ or it might mean that the user has decided to use \Verb+a+ as the braid parameter.
% So we examine the next token for a \Verb+t+. 
%    \begin{macrocode}
\def\braid@maybe@locate{%
  \afterassignment\braid@@maybe@locate\let\braid@token=%
}
%    \end{macrocode}
% \end{macro}
%
% \begin{macro}{\braid@@maybe@locate}
% This is where we test for \Verb+t+ and act appropriately.
%    \begin{macrocode}
\def\braid@@maybe@locate{%
  \let\braid@next=\braid@handle
  \ifx\braid@token t
   \let\braid@next=\braid@find@location
  \fi
  \braid@next%
}
%    \end{macrocode}
% \end{macro}
%
% \begin{macro}{\braid@find@location}
% This macro starts us looking for a coordinate.
%    \begin{macrocode}
\def\braid@find@location{%
  \afterassignment\braid@@find@location\let\braid@token=%
}
%    \end{macrocode}
% \end{macro}
%
% \begin{macro}{\braid@@find@location}
% This is the test for the start of a coordinate.
% If we get a \Verb+(+ that means we've reached the coordinate.
% A space means ``carry on''.
% Anything else is a (non-fatal) error.
%    \begin{macrocode}
\def\braid@@find@location{%
  \let\braid@next=\braid@location@error
  \ifx\braid@token(%)
   \let\braid@next=\braid@locate
  \else
  \ifx\braid@token\@sptoken
   \let\braid@next=\braid@find@location
  \fi
  \fi
  \braid@next%
}
%    \end{macrocode}
% \end{macro}
%
% \begin{macro}{\braid@location@error}
% This is our error message for not getting an error
%    \begin{macrocode}
\def\braid@location@error{%
  \errmessage{Could not figure out location for braid}%
  \braid@process@start%
}
%    \end{macrocode}
% \end{macro}
%
% \begin{macro}{\braid@locate}
% If we reached a \Verb+(+ when looking for a coordinate, everything up to the next \Verb+)+ is that coordinate.
%    \begin{macrocode}
\def\braid@locate#1){%
  \braid@relocate{(#1)}%
}
%    \end{macrocode}
% \end{macro}
%
% \begin{macro}{\braid@relocate}
% This is the macro that actually does the relocation.
%    \begin{macrocode}
\def\braid@relocate#1{%
  \path #1;
  \pgfgetlastxy{\braid@loc@x}{\braid@loc@y}
  \braid@tx=0pt
  \braid@ty=0pt
  \advance\braid@tx by \braid@loc@x
  \advance\braid@ty by \braid@loc@y
  \advance\braid@tx by -\braid@width
  \pgftransformshift{\pgfqpoint{\braid@tx}{\braid@ty}}
  \braid@process@start%
}
%    \end{macrocode}
% \end{macro}
%
% \begin{macro}{\braid@assign@name}
% This macro saves our name.
%    \begin{macrocode}
\def\braid@assign@name#1){%
  \def\braid@name{#1}%
  \braid@process@start%
}
%    \end{macrocode}
% \end{macro}
%
% \begin{macro}{\braid@process@options}
% The intention of this macro is to allow setting of style options mid-braid.
% (At present, this wouldn't make a lot of sense.)
%    \begin{macrocode}
\def\braid@process@options#1]{%
    \tikzset{#1}%
  \braid@process%
}
%    \end{macrocode}
% \end{macro}
%
% \begin{macro}{\braid@sup}
% This handles braid elements of the form \Verb+a^{-1}_2+.
%    \begin{macrocode}
\def\braid@sup#1_#2{\g@addto@macro\braid@label{_{#2}^{#1}}\braid@add@crossing{#2}{#1}}
%    \end{macrocode}
% \end{macro}
%
% \begin{macro}{\braid@sub}
%    \begin{macrocode}
% This handles braid elements of the form \Verb+a_1+ or \Verb+a_1^{-1}+.
\def\braid@sub#1{\@ifnextchar^{\braid@@sub{#1}}{\g@addto@macro\braid@label{_{#1}}\braid@add@crossing{#1}{1}}}
%    \end{macrocode}
% \end{macro}
%
% \begin{macro}{\braid@@sub}
% Helper macro for \Verb+\braid@sub+.
%    \begin{macrocode}
\def\braid@@sub#1^#2{\g@addto@macro\braid@label{_{#1}^{#2}}\braid@add@crossing{#1}{#2}}
%    \end{macrocode}
% \end{macro}
%
% \begin{macro}{\braid@ne}
% Remember what \Verb+1+ looks like for testing against.
%    \begin{macrocode}
\def\braid@ne{1}
%    \end{macrocode}
% \end{macro}
%
% \begin{macro}{\braid@add@crossing}
% This is the macro which add the crossing to the current list of strands.
% The strands are stored as \emph{soft paths} (see the TikZ/PGF documentation).
% So this selects the right strands and then extends them according to the crossing type.
%    \begin{macrocode}
\def\braid@add@crossing#1#2{%
  \edef\braid@crossing@type{#2}
  \edef\braid@this@strand{#1}
  \pgfmathtruncatemacro{\braid@next@strand}{#1+1}

% increment the counter, if requested
  \ifbraid@increase@level
  \stepcounter{braid@level}
  \fi
% request increment next time
  \braid@increase@leveltrue

% Coordinates of crossing
  \braid@tx=\braid@this@strand\braid@width
  \braid@ty=\value{braid@level}\braid@height
  \advance\braid@ty by \braid@eh
  \braid@nx=\braid@tx
  \braid@ny=\braid@ty
  \advance\braid@nx by \braid@width
  \advance\braid@ny by \braid@height
  \braid@cy=\braid@ty
  \braid@dy=\braid@ny
  \advance\braid@cy by .5\braid@height
  \advance\braid@dy by -.5\braid@height
  \advance\braid@ty by .05\braid@height
  \advance\braid@ny by -.05\braid@height

% Try to find a starting point for the strand ending here
  \expandafter\let\expandafter\braid@this@path@origin\csname braid@strand@\braid@this@strand @origin\endcsname

\ifx\braid@this@path@origin\relax
  % Haven't seen this strand before, so initialise it
  % Record the initial position of the strand
  \let\braid@this@path@origin\braid@this@strand
  % start a new soft path
  \pgfsyssoftpath@setcurrentpath{\@empty}
  \pgfpathmoveto{\pgfpoint{\braid@tx}{0pt}}
  % save the path
  \pgfsyssoftpath@getcurrentpath{\braid@this@path}
  \else
  \expandafter\let\expandafter\braid@this@path\csname braid@strand@\braid@this@path@origin\endcsname
  \fi

% Try to find a starting point for the next strand ending here
  \expandafter\let\expandafter\braid@next@path@origin\csname braid@strand@\braid@next@strand @origin\endcsname

  \ifx\braid@next@path@origin\relax
  % Haven't seen this strand before, so initialise it
  % Record the initial position of the strand
  \let\braid@next@path@origin\braid@next@strand
  % start a new soft path
  \pgfsyssoftpath@setcurrentpath{\@empty}
  \pgfpathmoveto{\pgfpoint{\braid@nx}{0pt}}
  % save the path
  \pgfsyssoftpath@getcurrentpath{\braid@next@path}
  \else
  \expandafter\let\expandafter\braid@next@path\csname braid@strand@\braid@next@path@origin\endcsname
  \fi

  % Start with the first path
  \pgfsyssoftpath@setcurrentpath{\braid@this@path}
  % Draw a line down to the current level
  \pgfpathlineto{\pgfqpoint{\braid@tx}{\braid@ty}}
  % Curve across to the next position
\ifx\braid@crossing@type\braid@over@cross
\pgfpathcurveto{\pgfqpoint{\braid@tx}{\braid@cy}}{\pgfqpoint{\braid@nx}{\braid@dy}}{\pgfqpoint{\braid@nx}{\braid@ny}}
\else
\pgfpathcurvebetweentimecontinue{0}{.4}{\pgfqpoint{\braid@tx}{\braid@ty}}{\pgfqpoint{\braid@tx}{\braid@cy}}{\pgfqpoint{\braid@nx}{\braid@dy}}{\pgfqpoint{\braid@nx}{\braid@ny}}
\pgfpathcurvebetweentime{.6}{1}{\pgfqpoint{\braid@tx}{\braid@ty}}{\pgfqpoint{\braid@tx}{\braid@cy}}{\pgfqpoint{\braid@nx}{\braid@dy}}{\pgfqpoint{\braid@nx}{\braid@ny}}
\fi
    % Save the path
  \pgfsyssoftpath@getcurrentpath{\braid@this@path}

  % Now do the same with the second path
  \pgfsyssoftpath@setcurrentpath{\braid@next@path}
  % Draw a line down to the current level
  \pgfpathlineto{\pgfqpoint{\braid@nx}{\braid@ty}}
  % Curve across to the previous position
\ifx\braid@crossing@type\braid@over@cross
\pgfpathcurvebetweentimecontinue{0}{.4}{\pgfqpoint{\braid@nx}{\braid@ty}}{\pgfqpoint{\braid@nx}{\braid@cy}}{\pgfqpoint{\braid@tx}{\braid@dy}}{\pgfqpoint{\braid@tx}{\braid@ny}}
\pgfpathcurvebetweentime{.6}{1}{\pgfqpoint{\braid@nx}{\braid@ty}}{\pgfqpoint{\braid@nx}{\braid@cy}}{\pgfqpoint{\braid@tx}{\braid@dy}}{\pgfqpoint{\braid@tx}{\braid@ny}}
\else
  \pgfpathcurveto{\pgfqpoint{\braid@nx}{\braid@cy}}{\pgfqpoint{\braid@tx}{\braid@dy}}{\pgfqpoint{\braid@tx}{\braid@ny}}
\fi
  % Save the path
  \pgfsyssoftpath@getcurrentpath{\braid@next@path}

% Now save the paths again
  \expandafter\let\csname braid@strand@\braid@this@path@origin \endcsname\braid@this@path
  \expandafter\let\csname braid@strand@\braid@next@path@origin \endcsname\braid@next@path

  % Now update the origins
  \expandafter\let\csname braid@strand@\braid@this@strand @origin\endcsname\braid@next@path@origin
  \expandafter\let\csname braid@strand@\braid@next@strand @origin\endcsname\braid@this@path@origin

% increment the strand counter, if necessary
  \pgfmathparse{\value{braid@strands} < \braid@next@strand ? "\noexpand\setcounter{braid@strands}{\braid@next@strand}" : ""}
  \pgfmathresult
  \braid@process%
}
%    \end{macrocode}
% \end{macro}
%
% \begin{macro}{\braid@floors@trim}
% The list of floors, if given, will start with a superfluous comma.
% This removes it.
%    \begin{macrocode}
\def\braid@floors@trim,{}
%    \end{macrocode}
% \end{macro}
%
% \begin{macro}{\braid@render@floor}
% This is the default render for floors: it draws a rectangle.
%    \begin{macrocode}
\def\braid@render@floor{%
    \draw (\floorsx,\floorsy) rectangle (\floorex,\floorey);
}
%    \end{macrocode}
% \end{macro}
%
% \begin{macro}{\braid@render}
% This is called at the end of the braid and it renders the braids and floors according to whatever has been built up up to now.
%    \begin{macrocode}
\def\braid@render{
    \ifx\braid@floors\@empty
    \else
    \pgfsys@beginscope
    \pgfsyssoftpath@setcurrentpath{\empty}
    \edef\braid@floors{\expandafter\braid@floors@trim\braid@floors}
    \braid@tx=\braid@width
    \advance\braid@tx by \braid@eh
    \braid@nx=\value{braid@strands}\braid@width
    \advance\braid@nx by -\braid@eh
    \foreach \braid@f in \braid@floors {
      \pgfsys@beginscope
      \braid@ty=\braid@f\braid@height
      \advance\braid@ty by \braid@eh
      \advance\braid@ty by \braid@height
      \braid@ny=\braid@ty
      \advance\braid@ny by \braid@height
      \edef\floorsx{\the\braid@tx}
      \edef\floorsy{\the\braid@ty}
      \edef\floorex{\the\braid@nx}
      \edef\floorey{\the\braid@ny}
      \let\tikz@options=\pgfutil@empty
% Load general floor style options
    \expandafter\tikzset\expandafter{\braid@floors@style}
% Load any style options specific to this floor
      \pgfmathtruncatemacro{\braid@ff}{\braid@f+2}
    \expandafter\let\expandafter\braid@floor@style\csname braid@options@floor@\braid@ff\endcsname
    \ifx\braid@floor@style\relax
    \else
    \expandafter\tikzset\expandafter{\braid@floor@style}
    \fi
\tikz@options
      \braid@render@floor
\pgfsys@endscope
    }
    \pgfsys@endscope
    \fi
  \stepcounter{braid@level}
  \foreach \braid@k in {1,...,\value{braid@strands}} {
% Start a local scope to ensure we don't mess with other braids
    \pgfsys@beginscope
% Default is to draw each braid
    \tikz@mode@drawtrue%
    \let\tikz@mode=\pgfutil@empty
    \let\tikz@options=\pgfutil@empty
% (x,y) coordinates of bottom of strand
    \braid@tx=\braid@k\braid@width
    \braid@ty=\value{braid@level}\braid@height
    \advance\braid@ty by 2\braid@eh
% Try to find the starting point of this strand
    \expandafter\let\expandafter\braid@path@origin\csname braid@strand@\braid@k @origin\endcsname
    \ifx\braid@path@origin\relax
% If that doesn't exist, we'll just draw a straight line
% so we move to the top of the current position
    \pgfsyssoftpath@setcurrentpath{\@empty}
    \pgfpathmoveto{\pgfqpoint{\braid@tx}{0pt}}
    \let\braid@path@origin\braid@k
    \else
% If the path does exist, we load it
    \expandafter\let\expandafter\braid@path\csname braid@strand@\braid@path@origin\endcsname
    \pgfsyssoftpath@setcurrentpath{\braid@path}
    \fi
% Extend the path to the bottom
    \pgflineto{\pgfqpoint{\braid@tx}{\braid@ty}}
% Load common style options
    \expandafter\tikzset\expandafter{\braid@style}
% Load any style options specific to this strand
    \expandafter\let\expandafter\braid@style\csname braid@options@strand@\braid@path@origin\endcsname
    \ifx\braid@style\relax
    \else
    \expandafter\tikzset\expandafter{\braid@style}
    \fi
\braid@options
    \tikz@mode
    \tikz@options
      \edef\tikz@temp{\noexpand\pgfusepath{%
          \iftikz@mode@draw draw\fi%
      }}%
      \tikz@temp
\ifx\braid@name\pgfutil@empty
\else
% label the ends of the strand
\coordinate (\braid@name-\braid@path@origin-e) at (\braid@tx,\braid@ty);
\coordinate (\braid@name-rev-\braid@k-e) at (\braid@tx,\braid@ty);
\braid@nx=\braid@path@origin\braid@width
\coordinate (\braid@name-\braid@path@origin-s) at (\braid@nx,0pt);
\coordinate (\braid@name-rev-\braid@k-s) at (\braid@nx,0pt);
\fi
% Close the scope
   \pgfsys@endscope
  }
    \ifx\braid@name\pgfutil@empty
    \else
    \braid@tx=\value{braid@strands}\braid@width
    \braid@ty=\value{braid@level}\braid@height
    \advance\braid@ty by 2\braid@eh
    \advance\braid@tx by \braid@width
    \braid@tx=.5\braid@tx
    \braid@ty=.5\braid@ty
    \coordinate (\braid@name) at (\braid@tx,\braid@ty);
    \fi
    \pgfsys@endscope
  \endgroup}
%    \end{macrocode}
% \end{macro}
%
% \begin{macro}{\braid@start}
% This starts off the braid, initialising a load of stuff.
%    \begin{macrocode}
\def\braid@start#1{%
      \pgfsys@beginscope
      \setcounter{braid@level}{-1}%
      \let\braid@label\@empty
      \let\braid@floors\@empty
      \let\braid@name\empty
      \pgfkeys{/pgf/braid/.cd,#1}
      \let\braid@options\tikz@options
      \setcounter{braid@strands}{\pgfkeysvalueof{/pgf/braid/number of           strands}}%
    \braid@width=\pgfkeysvalueof{/pgf/braid/width}
    \braid@height=\pgfkeysvalueof{/pgf/braid/height}
    \braid@eh=\pgfkeysvalueof{/pgf/braid/border height}
    \braid@height=-\braid@height
    \braid@eh=-\braid@eh
    \braid@increase@leveltrue
    \braid@process@start
}
%    \end{macrocode}
% \end{macro}
%
% These are the lengths we'll use as we construct the braid
%    \begin{macrocode}
\newdimen\braid@width
\newdimen\braid@height
\newdimen\braid@tx
\newdimen\braid@ty
\newdimen\braid@nx
\newdimen\braid@ny
\newdimen\braid@cy
\newdimen\braid@dy
\newdimen\braid@eh
%    \end{macrocode}
%
% An if to decide whether or not to step to the next level or not
%    \begin{macrocode}
\newif\ifbraid@increase@level
%    \end{macrocode}
%
% Some initial values
%    \begin{macrocode}
\let\braid@style\pgfutil@empty
\let\braid@floors@style\pgfutil@empty
\def\braid@over@cross{1}
%    \end{macrocode}
%
% Counters to track the strands and the levels.
%    \begin{macrocode}
\newcounter{braid@level}
\newcounter{braid@strands}
%    \end{macrocode}
%
% All the keys we'll use.
%    \begin{macrocode}
\pgfkeys{
    /tikz/braid/.search also={/pgf},
    /pgf/braid/.search also={/pgf,/tikz},
    /pgf/braid/.cd,
    number of strands/.initial=0,
    height/.initial=1cm,
    width/.initial=1cm,
    border height/.initial=.25cm,
    name/.code={%
      \def\braid@name{#1}%
    },
    at/.code={%
      \braid@relocate{#1}%
    },
    floor command/.code={%
      \def\braid@render@floor{#1}%
    },
    style strand/.code 2 args={%
      \expandafter\def\csname braid@options@strand@#1\endcsname{#2}
    },
    style floor/.code 2 args={%
      \expandafter\def\csname braid@options@floor@#1\endcsname{#2}
    },
    style floors/.code={%
      \def\braid@floors@style{#1}
    }
}
%    \end{macrocode}
% \iffalse
%</package>
% \fi
%
% \Finale

\endinput
